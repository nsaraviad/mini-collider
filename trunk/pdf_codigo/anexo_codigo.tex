\documentclass[10pt, a4paper,english,spanish]{article}
\usepackage{subfig}
\parindent=0 pt
\parskip=11 pt
\usepackage[width=15.5cm, left=3cm, top=2.5cm, height= 24.5cm]{geometry}
\usepackage{amsmath}
\usepackage{amsfonts}
\usepackage{amssymb}
\usepackage{fancyhdr}
\usepackage[T1]{fontenc}
\usepackage[spanish]{babel}
\usepackage[utf8]{inputenc}
\usepackage[pdftex]{graphicx}
\usepackage{hyperref}
\usepackage{minted}
\usepackage{caratula}

\usepackage{sectsty}
\sectionfont{\large}

\usepackage{fancyhdr}
\pagestyle{fancyplain}
\lhead{Mini-collider}
\rhead{Grupo 11}

\begin{document}

\materia{Teoría de lenguajes}
\titulo{Anexo código mini-collider}
\grupo{Grupo: 11}

\integrante{Calderini, Nicolás}{820/10}{calderini.nicolas@gmail.com}
\integrante{Hernández, Santiago}{48/11}{santi-hernandez@hotmail.com}
\integrante{Marasca, Dardo}{227/07}{dmarasca@yahoo.com.ar}
\integrante{Saravia, Nicolás}{905/04}{nicolasaravia@yahoo.com}
 
\maketitle
\newpage
\thispagestyle{empty}
\mbox{}
\newpage
\parskip 2pt
\parindent 20pt

\setcounter{page}{1}
\tableofcontents



\noindent \newline \newline El código entregado se encuentra dividido en módulos con la siguiente estructura:
\begin{verbatim}
+--- mini_collider.py
+--- minicollider
|    \--- lexer.py
|    \--- mixer.py
|    \--- parser.py
|    \--- test
|         \--- test_mixer.py
|         \--- test_parser.py
\end{verbatim}

\newpage

\newpage
\section{mini\_collider.py}
\inputminted[linenos, tabsize=4]{python}{../mini_collider.py}

\newpage
\section{lexer.py}
\inputminted[linenos, tabsize=4]{python}{../minicollider/lexer.py}

\newpage
\section{mixer.py}
\inputminted[linenos, tabsize=4]{python}{../minicollider/mixer.py}

\newpage
\section{parser.py}
\inputminted[linenos, tabsize=4]{python}{../minicollider/parser.py}

\newpage
\section{test\_mixer.py}
\inputminted[linenos, tabsize=4]{python}{../minicollider/test/test_mixer.py}

\newpage
\section{test\_parser.py}
\inputminted[linenos, tabsize=4]{python}{../minicollider/test/test_parser.py}



\end{document}




















